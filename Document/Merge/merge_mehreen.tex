%%%%%%%%%%%%%%%%%%%%%%%%%%%%%%%%%%%%%%%%%%%%%%%%%%%%%%%%%%%%%%%
%
% Welcome to Overleaf --- just edit your LaTeX on the left,
% and we'll compile it for you on the right. If you give
% someone the link to this page, they can edit at the same
% time. See the help menu above for more info. Enjoy!
%
%%%%%%%%%%%%%%%%%%%%%%%%%%%%%%%%%%%%%%%%%%%%%%%%%%%%%%%%%%%%%%%
% \documentclass{article} % Loads settings for the document layout
% \usepackage{mathtools}
% \usepackage{listings}
% \usepackage{color}
% 
% \definecolor{dkgreen}{rgb}{0,0.6,0}
% \definecolor{gray}{rgb}{0.5,0.5,0.5}
% \definecolor{mauve}{rgb}{0.58,0,0.82}
% 
% \lstset{frame=tb,
%   language=Python,
%   aboveskip=3mm,
%   belowskip=3mm,
%   showstringspaces=false,
%   columns=flexible,
%   basicstyle={\small\ttfamily},
%   numbers=none,
%   numberstyle=\tiny\color{gray},
%   keywordstyle=\color{blue},
%   commentstyle=\color{dkgreen},
%   stringstyle=\color{mauve},
%   breaklines=true,
%   breakatwhitespace=true,
%   tabsize=3
% }
% % Preambel
% 
% % The following settings are used for title generation and will show up in the
% % main document where the \maketitle command is set.
% \title{Title of my document}
% \date{2013-09-01}
% \author{John Doe}
% 
% % Main document
% 
% \begin{document} % The document starts here
% 
% \maketitle % Creates the titlepage
% \pagenumbering{gobble} % Turns off page numbering
% \newpage % Starts a new page
% \pagenumbering{arabic} % Turns on page numbering
% 
\section{Motivation}

 User modeling, adaptation, and personalization techniques have hit the mainstream. The explosion of social network websites, on-line user-generated content platforms, and the tremendous growth in computational power of mobile devices are generating incredibly large amounts of user data, and an increasing desire of users to "personalize" (their desktop, e-mail, news site, phone).

 The potential value of personalization has become clear both as a commodity for the benefit or enjoyment of end-users, and as an enabler of new or better services –a strategic opportunity to enhance and expand businesses.

 An exciting characteristic of recommender systems is that they draw the interest of industry and businesses while posing very interesting research and scientific challenges.

 In spite of significant progress in the research community, and industry efforts to bring the benefits of new techniques to end-users, there are still important gaps that make personalization and adaptation difficult for users. Research activities still often focus on narrow problems, such as incremental accuracy improvements of current techniques, sometimes with ideal hypotheses, or tend to overspecialize on a few applicative problems (typically TV or movie recommenders –sometimes simply because of the availability of data). This restrains de facto the range of other applications where personalization technologies might be useful as well.

 Thus, we may have reached a good point to take a step back to seek perspective in the research done in recommender systems. This workshop contrives for a new uptake on past experiences and lessons learned. We propose an analytic outlook on new research directions, or ones that still require substantial research, with a special focus on their practical adoption in working applications, and the barriers to be met in this path.






% \end{document} % The document ends here
