\section{Motivation}

 User modeling, adaptation and personalization methods have reached the mainstream. The tremendous growth of social networking websites and the computational power of mobile devices are generating extremely huge amounts of user data, and an increasing need of users to "personalize" their e-mail, phone etc.

 The potential value of personalization is now clear both as a commodity for the benefit of end-users and as an enabler of better (or new) services – a tactical opportunity to expand and improve businesses.

 The main characteristic of recommender systems is that they attract the interest of industry and businesses while posing extremely interesting challenges.

 In spite of noteworthy progress in the research community and the efforts of the industry to provide the end users the the benefits of new techniques, there are still many important gaps that make personalization and adaptation challenging for users. Research activities still focus on narrow problems, such as incremental accuracy improvements of current techniques or on a few applicative problems. This confines the range of other applications where personalization technologies might be useful as well.

 Thus, we have come to a good point where we can take a step back to obtain a perspective in the research done in recommender systems.
