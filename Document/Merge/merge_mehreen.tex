\section{Motivation}

 User modeling, adaptation, and personalization techniques have hit the mainstream. The explosion of social network websites, on-line user-generated content platforms, and the tremendous growth in computational power of mobile devices are generating incredibly large amounts of user data, and an increasing desire of users to "personalize" (their desktop, e-mail, news site, phone).

 The potential value of personalization has become clear both as a commodity for the benefit or enjoyment of end-users, and as an enabler of new or better services – a strategic opportunity to enhance and expand businesses.

 An exciting characteristic of recommender systems is that they draw the interest of industry and businesses while posing very interesting research and scientific challenges.

 In spite of significant progress in the research community, and industry efforts to bring the benefits of new techniques to end-users, there are still important gaps that make personalization and adaptation difficult for users. Research activities still often focus on narrow problems, such as incremental accuracy improvements of current techniques, sometimes with ideal hypotheses, or tend to overspecialize on a few applicative problems (typically TV or movie recommenders - sometimes simply because of the availability of data). This restrains the range of other applications where personalization technologies might be useful as well.

 Thus, we may have reached a good point to take a step back to seek perspective in the research done in recommender systems.