\section{Motivation}

 User modeling, adaptation and personalization methods have reached the mainstream. The growth of social networking websites and the computational power of mobile devices are generating huge amounts of user data and increasing the need of users to ``personalize" their e-mail, phone etc.

 The potential value of personalization is now clear both as a commodity for the benefit of end-users and as an enabler of better (or new) services. It serves as a tactical opportunity to expand and improve businesses.

 The main characteristic of recommender systems is that they attract the interest of industry and businesses while posing extremely interesting challenges.

 In spite of noteworthy progress in the research community and the efforts of the industry to provide the end users with the benefits of new techniques, there are still many important gaps that make personalization and adaptation challenging for users. Research activities still focus on narrow problems, such as incremental accuracy improvements of current techniques or on a few applicative problems. This confines the range of other applications where personalization technologies might also be useful.

 Thus, we have come to a good point where we can take a step back to obtain a perspective in the research done in recommender systems. In this study we will be using MovieLens data set~\cite{movielens}.

\section{Introduction}
\label{sec:approaches}
There are two popular techniques used for developing a recommender system~\cite{nbrsurvey}. These are discussed in the following section.
\subsection{Content-based}
Content-based System is used to recommend an item to a user based upon a description of the item and a profile of users interests and other metadata. This metadata could be information like age, sex, demography etc. The content-based approach to recommendation has its roots in the information retrieval (IR) community, and employs many similar techniques. The system \textit{learns} to recommend items that are similar to the ones that the user liked in the past. 
\subsection{Collaborative Filtering}
 Collaborative filtering is the process of finding information using collaboration among multiple agents. Typically, a set of ``nearest neighbors" are found whose past ratings have a strong correlation with candidate user. Scores for unseen items are predicted based on a combination of the scores estimated from the nearest neighbors.

Main approaches for collaborative filtering:
\begin{itemize}
\item User based collaborative filtering  is a straightforward algorithmic interpretation of the core premise of collaborative filtering: find other users whose past rating behavior is similar to that of the current user and use their ratings on other items to predict what the current user will like.
\end{itemize}
\begin{itemize}
\item Item based collaborative filtering generates predictions by using the user’s own ratings for other items combined with those items’ similar to the target item.
\end{itemize}

