\documentclass{article}
\usepackage[parfill]{parskip}    		% Activate to begin paragraphs with an empty line rather than an indent

\usepackage{amsmath} % this package allows to create equations with "*" which removes the  default equation numbering.

\usepackage{graphicx} % required to make the image code work 

\usepackage{booktabs} % useful for creating tables

\usepackage{pgfplotstable} % Generates table from .csv

% \noappendixtables                % Don't have appendix tables
% \noappendixfigures               % Don't have appendix figures
\usepackage[font=bf]{caption}	 % to make the figure captions bold
\usepackage{siunitx}			 % for writing micro seconds
\usepackage{tabularx}			 % for creating table
\usepackage{multirow}			 % for tables
\usepackage{url}				 % to use \url in references.bib
\usepackage{algorithm}			 % algorithm code
\usepackage{algpseudocode}		 % pseudo code package
\usepackage{fixltx2e}
% \usepackage{hyperref}

% Setup siunitx:
\sisetup{
  round-mode          = places, % Rounds numbers
  round-precision     = 2, % to 2 places
}


\begin{document}
%\pagenumbering{gobble}
%
%\tableofcontents
%\newpage
%\pagenumbering{arabic}
%
%\section{Section}
%
%Hello World!
%
%\subsection{Subsection}
%
%Structuring a document is easy!
%
%\subsubsection{Subsubsection}
%
%More text.
%
%
%
%Some more text in the very next line.
%
%\paragraph{Paragraph}
%
%Some more text.
%
%\subparagraph{Subparagraph}
%
%Even more text.
%
%\section{Another section}
%
%\begin{equation*}
%f(x) = x^2
%\end{equation*}
%
%\begin{align*}
%f(x) &= x^2\\
%g(x) &= \frac{1}{x}\\
%F(x) &= \int^a_b \frac{1}{3}    \left(\frac{1}{x^3}\right)
%\end{align*}
%
%\begin{equation*}
%\left[
%\begin{matrix} % the equation* above is required for matrix environment to work. The matrix env works only inside another math env.
%1 & 0\\
%0 & 1
%\end{matrix}
%\right]
%\end{equation*}

%%%%%%%%%%%%%%%%%%%%%%%%%%%%%%%%%%%%%%%%%%%
\subsection{Idea of \textit{similarity}}
Let's say the users of a website like Imdb [TODO FIXME] rate the movies on a scale of 1 to 5, where a rating of 5 implies the user really liked the movie and a rating of 1 implies the opposite. This data can be stored in the from of a \textit{utility matrix} where each row of the matrix corresponds to a user and each column corresponds to a movie. Consider an example rating set shown in Table [TODO FIXME ] that can be represented in the form of utility matrix \textit{utility matrix} U as shown in equation [TODO FIXME]. This 4x5 matrix stores the ratings given by 4 users to a set of 5 movies. As expected, not every user would have rated each of the 5 movies. The user-movie combination for which a rating does not exist is indicated by a 0.
\begin{table*}
\centering
\begin{tabular}{|c|c|c|c|c|c|}
\hline
\textbf{} 			& \textbf{Matrix} & \textbf{Titanic} & \textbf{DieHard} & \textbf{ForrestGump} & \textbf{Wall-E}\\ 
\hline
\textbf{John} 		& 	 	5		  & 		1		 & 				    &   	2			   & 		2		\\ 
\hline
\textbf{Lucy} 		& 	 	1		  & 		5		 & 		2		    &   	5			   & 		5		\\ 
\hline
\textbf{Eric} 		& 	 	2		  & 				 & 		3		    &   	5			   & 		4		\\ 
\hline
\textbf{Diane} 		& 	 	4		  & 		3		 & 		5		    &   	3			   & 				\\ 
\hline
\end{tabular}
\caption{Example ratings}
\label{table:example_ratings1}
\end{table*}

\begin{equation}
U = 
	\left[
	\begin{matrix} % the equation* above is required for matrix environment to work.
	5 & 1 & 0 & 2 & 0\\
	1 & 5 & 2 & 5 & 5\\
	2 & 0 & 3 & 5 & 4\\
	4 & 3 & 5 & 3 & 0
	\end{matrix}
	\right]
	\end{equation}
	
\textbf{
EXERCISE: Look at the ratings of user \textit{Eric} in Table \ref{table:example_ratings1}. Also look at the ratings given by other users in the table. Can you guess which user(s) are similar to user \textit{Eric}?
}

\subsection{Similarity weight computation}
The example data set in Table \ref{table:example_ratings1} is very small in size for which finding users that are \textit{similar} to the user of interest might be possible. In real data sets, we need define a measure of \textit{similarity} that can be used to find the items/users that are \textit{similar} to each other. The computation of of the similarity weights is one of the most critical aspects of building \textit{neighborhood-based} recommendation systems, since it can have a significant impact on the performance and accuracy of a system.

\textbf{Cosine Similarity:} We apply the traditional notion of Cosine Vector (CV) similarity to find the similarity between two users \textit{u} and \textit{v}. If \textbf{x\textsubscript{u}} is vector of ratings r\textsubscript{ui} of a user \textit{u}, then the cosine similarity of users \textit{u} and \textit{v} can be calculated using the equation [TODO FIXME].
 \begin{equation}
CV(u,v) = cos(\textbf{x}_{u}, \textbf{x}_{v}) = \frac
{\sum_{i \in{I_{uv}}}^{}{r_{ui}r_{vi}}}
{\sqrt	{	\sum_{i \in{I_{u}}}^{}{r_{ui}^2}	\sum_{j \in{I_{v}}}^{}{r_{vj}^2}}		}
\end{equation}
where I\textsubscript{u} represents the set of ratings for the movies that have been rated by both users.

\textbf{
EXERCISE2: In the example data set of Table [TODO FIXME], find the users that are most similar to user \text{Eric}. You can use the function \textit{CosSimVecMatrix} provided to generate the similarity of user \textit{Eric} with other users.
}

Hopefully that was easy and your have the answer - users that seem most similar to \textit{Eric} are \textit{Lucy} and \textit{Diane}. That was easy. Let's look at the ratings of \textit{Diane} who seems to be \textit{similar} to \textit{Eric}. If we calculate the means of ratings of \textit{Eric} and \textit{Diane}, 3.5 and 3.75, and subtract the respective means from their respective ratings, the new ratings look like as shown in Table \ref{table:centered_ratings1}.
\begin{table*}
\centering
\begin{tabular}{|c|c|c|c|c|c|}
\hline
\textbf{} 			& \textbf{Matrix} & \textbf{Titanic} & \textbf{DieHard} & \textbf{ForrestGump} & \textbf{Wall-E}\\ 
\hline
\textbf{Eric} 		& 	 	-1.5	  & 		0		 & 		-0.5		&   	1.5			   & 		0.5		\\ 
\hline
\textbf{Diane} 		& 	 	0.25	  & 		-0.75	 & 		1.25		&   	-0.75		   & 		0		\\ 
\hline
\end{tabular}
\caption{Mean centered ratings}
\label{table:centered_ratings1}
\end{table*}

Positive values in Table \ref{table:centered_ratings1} represent a preference for the movie by the user since it's rating is more than the average, whereas a negative value implies a disliking for the movie. On analyzing these \textit{mean-centered ratings}, \textit{Eric} and \textit{Diane} don't seem to be so \textit{similar} anymore. In fact, they seem to have quite the opposite taste.

\textbf{Pearson Correlation Similarity:} A major flaw in using \textit{Cosine Similarity} to find the \textit{similarity} between users is the fact that \textit{Cosine Similarity} does not take into account the differences in the mean and variance of the ratings made by the users. \textit{Pearson Correlation} (PC) similarity is a popular measure that can be used to compare the ratings of users since it removes the effects of mean and variance in ratings while calculating the \textit{similarity} between users. The PC similarity between users \textit{u} and \textit{v} can be calculated using the equation [TODO FIXME].
 \begin{equation}
PC(u,v) = PC(\textbf{x}_{u}, \textbf{x}_{v}) = \frac
{\sum_{i \in{I_{uv}}}^{}{(r_{ui} - \bar{r_{u}})(r_{vi} - \bar{r_{v}})}}
{\sqrt	{	\sum_{i \in{I_{uv}}}^{}{(r_{ui} - \bar{r_{u}})^2}	\sum_{i \in{I_{uv}}}^{}{(r_{vj} - \bar{r_{v}})^2}}		}
\end{equation}

where rubar [TODO FIXME] represents the mean of the ratings given by user \textit{u}. Note that PC accounts for variance in ratings only for the ratings of the movies I\textsubscript{uv} that have been rated by both users \textit{u} and \textit{v}. The sign of similarity weight calculated using PC indicates whether the correlation between the two users is direct or inverse, and the magnitude of similarity weight (ranging from 0 to 1) represents the strength of correlation. These will be used to predict unknown ratings for a user using \textit{neighborhood based} methods later.

\textbf{
EXERCISE3: Calculate the Pearson Correlation (PC) similarity of \textit{Eric} with other users in Table \ref{table:example_ratings1}. You can use the function \textit{PCSimVecMatrix} provided with the lab. Which user(s) would you say is \textit{similar} to \textit{Eric}. You can do a similar analysis for comparing the Cosine similarities and Pearson Correlation similarities of other users.
}
%%%%%%%%%%%%%%%%%%%%%%%%%%%%%%%%%%%%%%%%%%%
%\begin{figure}	% need to run the latex twice to generate correct figure reference numbers.
%
%  \includegraphics[width=\linewidth]{WiscMadImage.jpg}
%  \caption{Where the magic happens}
%  \label{fig:photo1}
%\end{figure}
%
%Figure \ref{fig:photo1} shows the magical place.
%
%\begin{table}[h!]
%  \begin{center}
%    \caption{Caption for the table.}
%    \label{tab:table1}
%    \begin{tabular}{r|c||l}
%      1 & 2 & 3\\
%      \hline % creates a line between 2 rows
%      a & bd & c\\
%    \end{tabular}
%  \end{center}
%\end{table}
%
%
%\begin{table}[h!]
%  \begin{center}
%    \caption{A Nice Table!}
%    \label{tab:table2}
%    \begin{tabular}{|c|c|c|}
%      %\toprule
%      \hline
%      Some & actual & content\\
%      %\midrule
%      \hline
%      %\toprule
%      prettifies & the & content\\
%      as & well & as\\
%      using & the & booktabs package\\
%      %\bottomrule
%      \hline
%    \end{tabular}
%  \end{center}
%\end{table}
%
%\begin{table}[h!]
%  \begin{center}
%    \caption{Autogenerated table from .csv file.}
%    \label{table3}
%    \pgfplotstabletypeset[
%      multicolumn names, % allows to have multicolumn names
%      col sep=comma, % the seperator in our .csv file
%      display columns/0/.style={
%		column name=$Value 1$, % name of first column
%		column type={S},string type},  % use siunitx for formatting
%      display columns/1/.style={
%		column name=$Value 2$,
%		column type={S},string type},
%      every head row/.style={
%		before row={\toprule}, % have a rule at top
%		after row={
%			% \si{\ampere} & \si{\volt}\\ % the units seperated by & % ampere==>A and volt==>V
%			\si{Amperes} & \si{Volts}\\ % 
%			\midrule} % rule under units
%			},
%		every last row/.style={after row=\bottomrule}, % rule at bottom
%    ]{DummyTable.csv} % filename/path to file
%  \end{center}
%\end{table}
%
%\begin{table}[h!]
%  \begin{center}
%    \caption{Adding a column to previous table}
%    \label{table4}
%    \pgfplotstabletypeset[
%      multicolumn names, % allows to have multicolumn names
%      col sep=comma, % the seperator in our .csv file
%      display columns/0/.style={
%		column name=$Value 1$, % name of first column
%		column type={S},string type},  % use siunitx for formatting
%      display columns/1/.style={
%		column name=$Value 2$,
%		column type={S},string type},
%      display columns/2/.style={
%		column name=$Value 3$,
%		column type={S},string type},
%      every head row/.style={
%		before row={\toprule}, % have a rule at top
%		after row={
%			% \si{\ampere} & \si{\volt}\\ % the units seperated by & % ampere==>A and volt==>V
%			\si{Amperes} & \si{Volts} & \si{Newtons}\\ % 
%			\midrule} % rule under units
%			},
%		every last row/.style={after row=\bottomrule}, % rule at bottom
%    ]{DummyTable4.csv} % filename/path to file
%  \end{center}
%\end{table}


\end{document}